\documentclass[tikz, border=10pt]{standalone}
\usepackage{pgfplots}
\pgfplotsset{compat=1.18}
\usepackage{amsmath}

\begin{document}
	
	\begin{tikzpicture}
		\begin{axis}[
			no markers, 
			domain=-4:4, 
			samples=100,
			axis lines=left, 
			% ylabel direkt über der Achse positioniert
			ylabel={$f(x)$},
			every axis y label/.style={at={(axis description cs:0,1)}, anchor=south},
			xlabel={$x$}, 
			every axis x label/.style={at={(current axis.right of origin)}, anchor=west},
			height=7cm, % Höhe leicht erhöht für die Formel
			width=11cm,
			xtick={-2, 0, 2},
			xticklabels={$\mu-\sigma$, $\mu$, $\mu+\sigma$},
			ytick=\empty,
			enlargelimits=upper,
			clip=false
			]
			
			% Definition der Gauß-Funktion
			\pgfmathdeclarefunction{gauss}{3}{%
				\pgfmathparse{1/(#3*sqrt(2*pi))*exp(-((#1-#2)^2)/(2*#3^2))}%
			}
			
			% Die Kurve plotten (Beispiel: mu=0, sigma=1)
			\addplot [very thick, cyan!70!black] {gauss(x, 0, 1)};
			
			% Vertikale Linie für den Mittelwert (mu)
			\draw [dashed, gray] (axis cs:0,0) -- (axis cs:0,0.4) node[above, black] {$\mu$};
			
			% Hilfslinien für Standardabweichungen
			\draw [dotted, gray] (axis cs:1,0) -- (axis cs:1,0.24);
			\draw [dotted, gray] (axis cs:-1,0) -- (axis cs:-1,0.24);
			
		\end{axis}
	\end{tikzpicture}
	
\end{document}