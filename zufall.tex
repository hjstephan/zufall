\documentclass[a4paper,12pt]{article}
\usepackage[utf8]{inputenc}
\usepackage[german]{babel}
\usepackage{amsmath}
\usepackage{amssymb}
\usepackage{amsthm}
\usepackage{geometry}
\geometry{margin=2.5cm}

\theoremstyle{definition}
\newtheorem{definition}{Definition}
\newtheorem{beispiel}{Beispiel}
\newtheorem{satz}{Satz}

\title{Zufall in der Mathematik}
\author{Stephan Epp}
\date{\today}

\begin{document}
	
\maketitle

\tableofcontents
\newpage

\section{Einführung}

Zufall wird in der Mathematik beschrieben für diskrete und kontinuierliche Ereignisse. Diskrete Ereignisse sind dabei die natürlichen Ereignisse, die sich jeder gut vorstellen kann. Ein diskretes Ereignis ist zum Beispiel das Würfeln eines Würfels. Ein Würfel hat sechs Seiten. Dabei hat jeden Seite des Würfels eine Augenzahl, dass jede Augenzahl von $1, \ldots, 6$ auf jeweils einer Würfelseite abgebildet ist. Es stellt sich die Frage, wenn jemand den Würfel einmal würfelt, auf welche Augenzahl der Würfel fällt? Da es sechs Seiten gibt, kann es sein, dass der Würfel bei einem Wurf zufällig auf die Augenzahl $3$ fällt. Ist die $3$ auf dem Würfel deshalb bevorzugt, weil der Würfel beim Wurf auf die Augenzahl $3$ gefallen ist? Kommt es zufällig öfter vor, dass der Würfel auf die Augenzahl $3$ fällt? Nein, der Würfel hat sechs gleich große Seiten und keine der Seiten des Würfels ist durch die Art des Würfels bevorzugt.

\section{Grundlagen}

Um jede Unruhe über den Ausgang des Zufalls allen Beteiligten zu nehmen, wird der Zufall für diskrete Ereignisse mit Hilfe der Mathematik beschrieben und analysiert. 

\subsection{Definition}

Für das diskrete Ereignis, bei dem ein Würfel einmal geworfen wird, können folgende diskrete Ereignisse zufällig eintreten. Der Würfel kann auf die Augenzahl $1$ fallen, der Würfel kann auf die Augenzahl $2$ fallen, ..., der Würfel kann auf die Augenzahl $6$ fallen. Dabei ist jeder Fall des Würfels auf eine andere Augenzahl ein Ereignis. Es gibt damit also sechs Ereignisse beim Würfeln des Würfels mit sechs Seiten.

\section{Zusammenfassung}
\subsection{Ausblick}
	
\end{document}